\section{Zielsetzung}
\label{sec:Zielsetzung}

In diesem Versuch soll die Schallgeschwindigkeit in Acryl mit verschiedenen Ultraschallmessmethoden
bestimmt werden.

\section{Theorie}
\label{sec:Theorie}

\subsection{Schall im Medium} \label{sec:SIM}
Der in diesem Versuch verwendete Schall befindet sich im Frequenzbereich $\SI{20}{kHz}$ - $\SI{1}{GHz}$.
Schall in diesem Frequenzbereich wird als Ultraschall bezeichnet und ist vom menschlichen Gehöhr nicht wahrnehmbar. \newline
Schallwellen pflanzen sich in Gasen und Flüssigkeiten durch Druckschwankungen longitudinal fort. Es handelt sich bei Schall also um eine Longitudinalwelle. Für
die orts- und zeitabhängige Druckverteilung gilt:
\begin{equation}
  p(x,t)=p_0+v_0 Z \cos(\omega t - kx) \text{.} \label{eqn:pxt}
\end{equation}
\begin{center}
 \tiny {$Z\: \hat{=} \:\text{akustische Impedanz}=c\rho $, $c\: \hat{=} \: \text{Schallgewindigkeit}$, $\rho\: \hat{=} \: \text{Dichte des Mediums}$ )}
\end{center}
Eine weitere Eigenschaft von Schallwellen lässt sich aus Gleichung \eqref{eqn:pxt} entnehmen: Die Schallgewindigkeit ist abhängig vom Medium.
Für die Schallgewindigkeit in Flüssigkeiten lässt sich der Zusammenhang
\begin{equation}
c_\text{{Fl}}=\sqrt{\frac{1}{\kappa\rho}}\label{eq:Fl}
\end{equation}
\begin{center}
 \tiny {($\kappa \: \hat{=} \:\text{Kompressibilität der Flüssigkeit}$)}
\end{center}
aufstellen.
Aufgrund auftretender Schubspannungen können in Festkörpern auch Transversalwellen auftreten, weshalb für die Schallgeschwindigkeit im Festkörper das Elastizitätsmodul $E$ von wichtiger Bedeutung ist.
Mit diesem ergibt sich für die Schallgewindigkeit in Festkörpern der Zusammenhang
\begin{equation}
  c_\text{{Fe}}=\sqrt{\frac{E}{\rho}} \label{eq:Fe} \text{.}
\end{equation}
Jedoch ist nicht nur die Ausbreitungsgeschwindigkeit der Schallwelle vom Medium abhängig, sondern auch die abnehmende Intensität $I$ im Medium.
Für diese gilt:
\begin{equation}
  I(x)=I_0\cdot\mathrm{e}^{-\alpha x}\label{eq:I} \text{.}
\end{equation}
\begin{center}
  \tiny {($\alpha \: \hat{=} \:\text{Absorptionskoeffizient}$)}
\end{center}
Da Luft die Intensität von Schallwellen stark abschwächt, wird in diesem Versuch ein Kontaktmittel verwendet.
Geht die Schallwelle von einem Medium mit akustischer Impedanz $Z_1$ in ein Medium akustischer mit Impedanz $Z_2$ über, so wird ein Teil der Schallwelle reflektiert, während der andere Teil durch
die Grenzfläche transmittiert.
Für den Reflektionskoeffizienten gilt
\begin{equation}
R=\left(\frac{Z_1-Z_2}{Z_1+Z_2}\right)^2 \text{,} \label{eq:R}
\end{equation} während für den Transmissionskoeffizienten folglich
\begin{equation}
  T=1-R
\end{equation}
gilt.

\subsection{Erzeugung von Ultraschall und Anwendung}
\subsubsection{Erzeugung}
Ultraschall kann mit einem sogeannten piezoelektrischen Kristall erzeugt und gemessen werden.
Zum Erzeugen von Ultraschallwellen wird dieser in einem, mit Wechselstrom betriebenden, E-Feld platziert. Der Krsitall wird dort zu Schwingungen angeregt, wodurch Ultraschall entsteht.
Wenn die Anregungsfrequenz mit der Eigenfrequenz übereinstimmt, kommt es zur Resonanz, wodurch sehr große Schwingungsamplituden erreicht werden.
Der Kristall kann ebenso als Empfänger genutzt werden, da dieser auch durch Schall in Schwingungen versetzt werden kann.
\subsubsection{Anwendung}
Es gibt zwei grundlegende Techniken, mit denen, unter Einsatz von Ultraschall, Informationen über das zu untersuchende Objekt erhalten werden können.
Eine der Verfahrensweise ist die Durchschallungs-Methode. Bei dieser liegen Sender und Empfänger auf gegenüberliegenden Seiten der Probe.
Der Sender sendet Schallimpulse aus, die durch den zu untersuchenden Körper wandern und schließlich zum Empfänger gelangen.
Befinden sich Fehlstellen im Körper, so registriert der Empfänger eine verminderte Intensität.
So können zwar Fehlstellen registriert, aber nicht lokalisiert werden.
Um Fehlstellen zu lokalisieren, kann die zweite Verfahrensweise, das Impuls-Echo-Verfahren, verwendet werden.
Bei dieser befinden sich Sender und Empfänger auf der gleichen Seite der Probe. Trifft der ausgesendete Schallimpuls auf eine Fehlstelle, so wird ein Teil der Welle reflektiert
und gelangt zum Empfänger. Durch den Zusammenhang
\begin{equation}
s=\frac{1}{2}c t
\end{equation}
kann die Fehlstelle lokalisiert werden, da durch den Messvorgang Schallgeschwindigkeit und Zeit bekannt sind.
