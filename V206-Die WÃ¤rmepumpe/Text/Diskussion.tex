\section{Diskussion}

Wie in Kapitel \ref{sec:gut} zu sehen ist, ist die Abweichung zwischen der idealen und der realen
Güteziffer sehr groß. Dies kann daran liegen, dass
der Kompressor nicht, wie im idealen Fall angenommen, vollständig adiabatisch komprimieren kann.
Außerdem sind die Reservoire Plastikeimer und bieten somit keine gute Isolierung.
Desweiteren ist die Isolierung der Rohre nicht ideal.
Der Prozess ist nicht reversibel, obwohl dies im idealen Fall ebenfalls angenommen wird.
Es entstehen Energieverluste durch Reibung.
Die Skala der Manometer (besonders bei $p_b$) ist grob unterteilt und somit schwierig abzulesen.

Der geringe Wirkungsgrad steht in direkter Verbindung mit der Güteziffer und
lässt sich ebenfalls durch die oben genannten Gründe erklären.
