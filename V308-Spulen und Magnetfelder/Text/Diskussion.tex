\section{Diskussion}

Innerhalb der kurzen Spule ist das Magnetfeld inhomogen und hat sein Maximum ungefähr in der Mitte der Spule.
In der langen Spule hingegen bildet sich ein homogenes Magnetfeld aus, das etwas stärker als der Theoriewert ist.

Bei den Helmholtz-Spulen ist der Einfluss des Spulenabstandes gut zu erkennen.
Bei einem Abstand von $\SI{7}{\centi \meter}$ bildet sich ein homogenes Magnetfeld zwischen den beiden Spulen aus.
Die Stärke des Magnetfeldes weicht hier nur minimal vom theoretischen Wert ab.
Wird der Abstand vergrößert entsteht ein inhomogenes Magnetfeld mit einem Minimum genau zwischen den Spulen.
Die Theoriewerte an dieser Stelle, stimmen gut mit den Messungen überein.

Die Messwerte aus Kapitel \ref{sec:hys} ergeben eine klare Hysteresekurve aus der sich alle charakterisitschen
Größen gut ablesen lassen.

Eine mögliche Fehlerquelle bei der Aufnahme der Hysteresekurve ist, dass zu Beginn des Experiments die Remanenz
nicht ganz Null war. Außerdem lieferte die Hall-Sonde an einigen Stellen keine eindeutigen Messwerte.
Die Schwierigkeit bei den Helmholtz-Spulen und den Solenoiden besteht vor allem darin die Hall-Sonden richtig
auszurichten und zu verhindern, dass sie sich im Laufe der Messung verschieben.
