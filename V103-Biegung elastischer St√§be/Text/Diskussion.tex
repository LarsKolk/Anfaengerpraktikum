\section{Diskussion}
Die Literaturwerte\cite[95]{elast} der Elastizitätsmodule lauten:
\begin{align*}
  E_\text{Aluminium} &= 71 \si{\giga\pascal} \\
  E_\text{Messing} &= 98 \si{\giga\pascal}
\end{align*}
Daraus ergeben sich folgene Abweichungen:
\begin{align*}
\delta_\text{Aluminium} &= 4,21\%\\
\delta_\text{Messing,Einseitig} &= 2,18\%\\
\delta_\text{Messing,Beiseitig} &= 122,84\% %UNBEDINGT ERGÄNZEN!!!!
\end{align*}
Die Abweichungen für die einseitige Einspannung fallen mit $4,21\%$ und $2,18\%$ sehr gering aus und liegen innerhalb
der Messungenauigkeiten.
Trotz des analogen Vorgehens fällt der Fehler für die beidseitige Einspannung mit $122,84\%$ deutlich höher aus.
Dies kann folgene Gründe haben:
\begin{itemize}
  \item Der Aufsetzpunkt der Messuhr ist frei beweglich, sodass dieser sich um 180° um die eigene Achse gedreht haben kann,
  ohne, dass es den Experimentatoren aufgefallen ist.
  Dies ist nicht zu vernachlässigen, da eine solche Drehung große Unterschiede in den Messweten verursacht.
  \item Das System reagiert sehr empfindlich auf Stöße, was ebenfalls zu Abweichungen führen kann.
  \item Es kann ebenfalls sein,
  dass die extrem geringe Gesamtauslenkung des Stabes bei der beidseitigen Auflage (gerade mal $\SI{1,2}{\mm}$)
  nicht ausreicht, um den Elastizitätsmodul genau zu bestimmen.
\end{itemize}

Zusammenfassend lässt sich sagen, dass die Methode mit der einseitigen Einspannung genauere Werte liefert
und der beidseitigen Einspannung vorzuziehen ist.
