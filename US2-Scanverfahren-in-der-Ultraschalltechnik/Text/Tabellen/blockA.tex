\begin{table}[H]
   \centering
   \caption{Die mithilfe des A-Scans aufgenommenen und berechneten Werte}
   \label{tab:ascan}
   \makebox[1 \textwidth][c]{       %centering table
   \begin{tabular}{ c S S S S S S S }
 \toprule
 {Lochnummer} & {$d_\text{theo}\:/\: \mathrm{cm}$} & {$t_\text{oben}\:/\: \symup{\mu s}$} & {$s_\text{oben}\:/\: \mathrm{mm}$} &
 {$t_\text{unten}\:/\: \symup{\mu s}$} & {$s_\text{unten}\:/\: \mathrm{mm}$} & {$d\:/\: \mathrm{mm}$} & {Abweichung} \\
    \midrule
    3 & 0,6 & 46,34 & 61,25 & 11,00 & 13,01 & 5,73 & 4,48\% \\
    4 & 0,5 & 40,63 & 53,46 & 17,14 & 21,40 & 5,14 & 2,88\% \\
    5 & 0,4 & 35,23 & 46,09 & 23,91 & 30,64 & 3,27 & 18,15\% \\
    6 & 0,3 & 29,84 & 38,73 & 29,73 & 38,58 & 2,69 & 10,43\% \\
    7 & 0,2 & 23,70 & 30,35 & 35,55 & 46,53 & 3,12 & 56,19\% \\
    8 & 0,2 & 18,09 & 22,69 & 41,58 & 54,76 & 2,55 & 27,52\% \\
    9 & 0,2 & 12,17 & 14,61 & 47,19 & 62,41 & 2,97 & 48,68\% \\
    10 & 0,2 & 6,35 & 6,67 & 53,33 & 70,80 & 2,54 & 26,84\% \\
    11 & 0,9 & 42,22 & 55,63 & 12,49 & 15,05 & 9,32 & 3,57\% \\
    \bottomrule
  \end{tabular}
  }
\end{table}
