\section{Theorie \cite{sample}}

\textbf{\underline{Ziel:}}
Die Bestimmung des Elastizitätsmoduls unterschiedlicher Metallstäbe unter ein- und beidseitiger Einspannung.
\\
\\
Wirkt auf einen Körper eine Kraft $F$, so wirken auf ihn Spannungen $\sigma$.
Dabei ist die Spannung $\sigma$ als Kraft pro Fläche definiert.
Die senkrechte Komponente der Spannung wird als Normalspannung und die parallele Komponente als Schubspannung bezeichnet.
Das Hook'sche Gesetz beschreibt die Abhängigeit der Spannung $\sigma$ von dem Elastizitätsmodul $E$
und der relativen Längenänderung $\frac{\Delta L}{L}$:
\begin{equation}
  \sigma = E \cdot \frac{\Delta L}{L}.
\end{equation}
Im Experiment werden Biegungen ausgenutzt, da sich bereits bei geringen Kräften eine Verformung des Körpers erkennen lässt.
Die wirkende Kraft erzeugt ein Drehmoment, sodass
die unteren Schichten des Körpers gestaucht und oberen Schichten gestreckt werden.
Daher existiert in der Mitte des Körpers auch eine neutrale Faser.
Dies hat zur Folge, dass innere Normalspannungen auftreten.
Diese treten sowohl oberhalb als auch unterhalb der neutralen Faser auf und wirken entgegengesetzt.
Der Stab biegt sich, bis das gewichtsbedingte Drehmoment gleich dem spannungsbedingtem Drehmoment ist:
\begin{align}
  M_{\text{außen}} &= M_{\text{innen}} \\
  \implies F(L-x) &= \int_q y \sigma (y) dq
\end{align}
Dadurch ergibt sich für die einseitige Einspannung für die Auslenkung $D$:
\begin{equation}
  D(x)= \frac {F}{2 \cdot E \cdot I} \cdot (Lx^2-\frac{x^3}{3}).
  \label{eqn:ein}
\end{equation}
Dabei ist $L$ die eingespannte Länge des Stabes, $x$ die Entfernung des Messpunktes zum Einspannpunkt und $I$ das Flächenträgheitsmoment.
Bei beidseitiger Einspannung greift die Kraft in der Mitte des Stabes an und es ergibt sich für die Auslenkung $D$:
\begin{align}
D(x) &=  \frac {F}{48 \cdot E \cdot I} \cdot (3L^2x-4x^3), & \text{für} \,\,\, 0 \leq x \leq \frac{L}{2}.
\label{eqn:beid1}\\
D(x) &=  \frac {F}{48 \cdot E \cdot I} \cdot (4x^3-12Lx^2+9L^2x-L^3), & \text{für}\,\,\, \frac{L}{2} \leq x \leq L.
\label{eqn:beid2}
\end{align}
