\section{Diskussion}

Die in Kapitel \ref{sec:mol} bestimmten Molwärmen weichen stark von dem Dulong-Petitschen Gesetz ab und legen
den Schluss nahe, dass die klassische Methode nicht ausreicht, um die oszillatorische Bewegung der Atome zu
beschreiben.

Allerdings sind die berechneten Werte nicht sehr aussagekräftig, da bereits die im Vorfeld bestimmten
spezifischen Wärmekapazitäten der Probenkörper Abweichungen zwischen $15,27 \%$ und $38,06 \%$
von den jeweiligen Literaturwerten aufweisen.

Eine große Fehlerquelle bei der Bestimmung der spezifischen Wärmekapazitäten ist die Bestimmung der
Temperatur des Probenkörpers. Da dieser in einem Wasserbad aufgeheizt wird und nur die Wassertemperatur
gemessen werden kann, kann nicht die genaue Temperatur festgestellt werden.
Desweiteren liegt kein abgeschlossenes System vor und es wird Wärme vom Probenkörper und Kalorimeter an die
Umgebung abgegeben.
Außerdem ist es bei der Bestimmung der Mischtemperaturen nicht möglich, das Wasser umzurühren, sodass sich die
wärmere obere Wasserschicht nicht mit der kälteren unteren Schicht vermischt.
Dies führt zu der falschen Annahme, dass die vom Thermometer konstant angezeigt Temperatur auch wirklich die Mischtemperatur ist.
