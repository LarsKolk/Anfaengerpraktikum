\section{Diskussion}
Die Methoden, bei der die Gravitation (s.\ref{sec:grav}) und Präzession (s. \ref{sec:prae}) ausgenutzt werden,
waren mit einer prozentualen Abweichung von $3,19\%$ und $3,69\%$ etwas genauer, als die Methode,
die sich die Schwingungsdauer zu Nutze macht (s. \ref{sec:schw}). Da erhält man eine Abweichung von 6,54\%.
Bei der Betrachtung der Ergebnisse müssen aber auch die potentiellen Fehlerquellen beachtet werden.
Dazu gehören:

\begin{itemize}
  \item Bei \ref{sec:grav} war die Aluminiumstange leicht verbogen,
        wodurch die Gleichgewichtslage schwer zu erkennen war.
  \item Weitere Ungenauigkeiten bei \ref{sec:grav} können ungenaue Waagen und Messschieber sein.
  \item Eine Schwierigkeit bei \ref{sec:schw} ist die manuelle Zeitmessung, bei der sich der Experimentator
        auf seine Urteilsfähigkeit und Reflexe verlassen muss, die bekanntermaßen fehlerbehaftet sind.
  \item Trotz der Hilfe durch das Stroboskop bei \ref{sec:prae} ist es äußerst schwierig zu erkennen, wann die Kugel
        die richtige Frequenz erreicht hat.
  \item Auch wenn bei \ref{sec:prae} die Rotationsfrequenz so gewählt wird, dass sie möglichst langsam abfällt,
        so kann das Abfallen nicht vollständig verhindert werden und kann einen Einfluss auf das Experiment haben.
\end{itemize}
Zusammenfassend lässt sich sagen, dass alle drei Messmethoden gute Ergebnisse für das magnetische Moment liefern.
Trotz des größeren Fehlers in dieser Messreihe ist die Methode aus \ref{sec:schw} potentiell am genauesten,
da es möglich wäre eine automatisierte Zeitmessung einzuführen und somit der größte Teil der Fehlerquellen beseitigt wäre.
