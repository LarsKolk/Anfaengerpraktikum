\newpage
\section{Diskussion}

\textbf{\underline{Abweichungen:}}
\begin{itemize}
  \item[] $\delta_{R_\text{eff}} = 2,73 \%$
  \item[] $\delta_{T_\text{ex}} = 2,66 \%$
  \item[] $\delta_{R_\text{ap}} = 23,12 \%$
  \item[] $\delta_{q} = 6,55 \%$
  \item[] $\delta_{\nu_+ - \nu_-} = 51,73 \%$
  \item[] $\delta_{\nu_\text{res}} = 2,50 \%$
  \item[] $\delta_{\nu_1} = 7,13 \%$
  \item[] $\delta_{\nu_2} = 1,55 \%$
\end{itemize}

Allgemein lässt sich sagen, dass das Experiment überwiegend gute Ergebnisse liefert.
Die Abweichungen von $R_\text{eff}$, $T_\text{ex}$, $q$, $\nu_\text{res}$, $\nu_1$ und $\nu_2$
liegen mit unter $10 \%$ im Rahmen der Messungenauigkeiten.
Die einzigen Werte mit besonders hohen Abweichungen sind $R_\text{ap}$ und $\nu_+ - \nu_-$.
Beim aperiodischen Grenzfall lässt sich die Abweichung dadurch erklären,
dass aufgrund der begrenzten Auflösung des Oszilloskops und der daraus resultierenden Dicke,
der abgebildeten Linie, äußerst schwer zu erkennen ist wann das "Überschwingen" tatsächlich verschwindet.
Die Breite der Resonanzkurve ist besonders fehlerbehaftet,
da die Werte für $\nu_+$ und $\nu_-$ per Hand aus dem Graphen mit den Messwerten genommen werden müssen
und dies automatisch zu Fehlern bzw. Ungenauigkeiten beim Ablesen führt.
