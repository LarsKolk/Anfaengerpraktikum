\section{Diskussion}

Fast alle Messwerte befinden sich innerhalb der ersten beiden Fehlerintervalle und liegen somit im Rahmen der Messungenauigkeiten. Die Abweichungen von
den nominellen Werten sind auch bis auf die Breite des Doppelspalts in Kapitel \ref{sec:2dop} sehr klein. Allerdings ist die relative Abweichung
in Kapitel \ref{sec:2dop} mit $\SI{5,49}{\%}$ so gering, dass trotzdem von einer präzisen Bestimmung des Wertes ausgegangen werden kann.
Lediglich der Wert für den Abstand des Doppelspaltes in Kapitel \ref{sec:1dop} weicht mit $\SI{5}{}$ Fehlerintervallen stark
von seinem Theoriewert ab. Gründe hierfür können ein phasenweise anderer Dunkelstrom und Messwertverzerrungen aufgrund des häufigen
Wechsels des Messbereiches sein.
Die Beugungsfigur eines Einzelspaltes stellt in Kapitel \ref{sec:1dop} und \ref{sec:2dop} wie erwartet die Einhüllende der Doppelspaltfunktion dar.
