\section{Auswertung}

\subsection{Indium \label{sec:ind}}
In Tabelle \ref{tab:ind} befinden sich die aufgenommenen Messwerte, bei denen der Nulleffekt von $\SI{0,25}{\per \s}$ bereits berücksichtigt wurde.
\begin{table}[H]
   \centering
   \caption{name}
   \label{tab:werte}
   \begin{tabular} { c S S S S S S S S S S S }
 \toprule
 {Lochnummer} & {$d_\text{gemessen}\:/\: \mathrm{cm}$} & {$t_\text{oben, A-Scan}\:/\: \symup{\mu s}$} & {$s_\text{oben, A-Scan}\:/\: \mathrm{mm}$} &
 {$t_\text{unten, A-Scan}\:/\: \symup{\mu s}$} & {$s_\text{unten, A-Scan}\:/\: \mathrm{mm}$} & {$d_\text{A-Scan}\:/\: \mathrm{mm}$} & {Abweichung 1} &
 {$t_\text{oben, B-Scan}\:/\: \symup{\mu s}$} & {$t_\text{unten, B-Scan}\:/\: \symup{\mu s}$} & {$d_\text{B-Scan}\:/\: \mathrm{mm}$} & {Abweichung 2} \\
    \midrule
    3 & 0,6 & 46,34 & 61,25 & 11,00 & 13,01 & 5,73 & 4,48\% & 45,93 & 10,74 & 6,65 & 10,83\% \\
    4 & 0,5 & 40,63 & 53,46 & 17,14 & 21,40 & 5,14 & 2,88\% & 40,37 & 16,85 & 5,89 & 17,83\% \\
    5 & 0,4 & 35,23 & 46,09 & 23,91 & 30,64 & 3,27 & 18,15\% & 34,81 & 23,15 & 4,88 & 22,01\% \\
    6 & 0,3 & 29,84 & 38,73 & 29,73 & 38,58 & 2,69 & 10,43\% & 29,44 & 29,44 & 3,62 & 20,56\% \\
    7 & 0,2 & 23,70 & 30,35 & 35,55 & 46,53 & 3,12 & 56,19\% & 23,52 & 35,19 & 3,87 & 93,47\% \\
    8 & 0,2 & 18,09 & 22,69 & 41,58 & 54,76 & 2,55 & 27,52\% & 17,78 & 41,11 & 3,62 & 80,83\% \\
    9 & 0,2 & 12,17 & 14,61 & 47,19 & 62,41 & 2,97 & 48,68\% & 11,85 & 47,04 & 3,62 & 80,83\% \\
    10 & 0,2 & 6,35 & 6,67 & 53,33 & 70,80 & 2,54 & 26,84\% & 6,11 & 0,00 & - & - \\
    11 & 0,9 & 42,22 & 55,63 & 12,49 & 15,05 & 9,32 & 3,57\% & 41,67 & 12,22 & 10,44 & 16,02\% \\
    \bottomrule
  \end{tabular}
\end{table}

In Graph \ref{fig:ind} ist die Zerfallskurve des Indium-Zerfalls mit den Poisson-Fehlern $\sqrt{N}$ aufgetragen.
\begin{figure}[H]
  \centering
  \includegraphics[width=\textwidth]{Plots/ind.pdf}
  \caption{Zerfallskurve von $\ce{^{116}In}$}
  \label{fig:ind}
\end{figure}

Eine lineare Regression $f(x) = -a \cdot x + b$ liefert die Werte
\begin{align*}
  a &= \SI{2,52(10)e-4}{\per \s} \\
  b &= \SI{2,38(2)}{}
\end{align*}

Die Halbwertszeit des Indium-Isotops beträgt somit
\begin{equation*}
  \tau_{\ce{^{116}In}} = \frac{\ln(2)}{a} = \SI{45,76(179)}{\min}.
\end{equation*}

Der Fehler ergibt sich aus der Gauß'schen Fehlerfortpflanzung
\begin{equation}
  \symup{\Delta} \tau_{\ce{^{116}In}} = \sqrt{\left(-\frac{\ln{(2)}}{a^2} \cdot \symup{\Delta}a \right)^2}
\end{equation}

Der Theoriewert \cite{sample2} ist $\tau_{\ce{^{116}In}, \text{theo}} = \SI{54,29}{\min}$. Die Abweichung liegt bei $15,72 \%$.

Außerdem ergibt sich
\begin{equation*}
  N_0 \left(1 - e^{-\symup{\Delta}t a} \right) = \SI{10,83(23)}{\becquerel}.
\end{equation*}

\subsection{Rhodium \label{sec:rhod}}

In Tabelle \ref{tab:rhod} befinden sich die aufgenommenen Messwerte, bei denen der Nulleffekt von $\SI{0,25}{\per \s}$ bereits berücksichtigt wurde.
\begin{table}[H]
   \centering
   \caption{Aus Abbildung \ref{fig:herz} abgelesene Werte}
   \label{tab:herz}
   \begin{tabular} { S S }
 \toprule
 {$t\:/\: \mathrm{s}$} & {$h\:/\: \symup{\mu s}$} \\
    \midrule
     & 29,26 \\
    2,08 & 29,07 \\
    2,11 & 28,52 \\
    1,97 & 29,26 \\
    2,19 & 29,63 \\
    2,11 & 28,52 \\
    1,89 & 28,33 \\
    1,89 & 27,78 \\
    1,92 & 27,59 \\
    1,86 & 27,22 \\
    2,06 & 28,15 \\
    1,83 & 28,15 \\
    2,00 & 27,96 \\
    \bottomrule
  \end{tabular}
\end{table}


Zuerst wird die Zerfallskurve des langlebigen Zerfalls $\ce{^{104i}Rh}$ bestimmt.
In Abbildung \ref{fig:r1} sind die Werte für $t \geq \SI{468}{\s}$ aufgetragen.
\begin{figure}[H]
  \centering
  \includegraphics[width=\textwidth]{Plots/rhodi.pdf}
  \caption{Zerfallskurve von $\ce{^{104i}Rh}$}
  \label{fig:r1}
\end{figure}

Aus einer linearen Regression $f(x) = -a \cdot x + b$ mit
\begin{align*}
  a &= \SI{2,6(7)e-3}{\per \s} \\
  b &= \SI{4,6(119)}{}
\end{align*}

mit der Gauß'schen Fehlerfortpflanzung \eqref{eqn:err1}
\begin{equation}
  \symup{\Delta} \tau_{\ce{^{104i}Rh}} = \sqrt{\left(-\frac{\ln{(2)}}{a^2} \cdot \symup{\Delta}a \right)^2}
  \label{eqn:err1}
\end{equation}

ergibt sich die Halbwertszeit
\begin{equation*}
  \tau_{\ce{^{104i}Rh}} = \frac{\ln(2)}{a} = \SI{270(70)}{\s}.
\end{equation*}

Der Theoriewert \cite{sample3} liegt bei $\tau_{\ce{^{104i}Rh}, \text{theo}} = \SI{260}{\s}$, somit beträgt die Abweichung $3,85 \%$.

Beim kurzlebigen Zerfall $t \leq \SI{216}{\s}$ gilt
\begin{equation*}
  \ln{\left(N_{\ce{^{104}Rh}}\right)} = \ln{\left(N_\text{ges} - N_{\ce{^{104i}Rh}}\right)} = -c \cdot x + d.
\end{equation*}

Dies ist in Abbildung \ref{fig:r2} dargestellt.
\begin{figure}[H]
  \centering
  \includegraphics[width=\textwidth]{Plots/rhod.pdf}
  \caption{Zerfallskurve von $\ce{^{104}Rh}$}
  \label{fig:r2}
\end{figure}

Für die Parameter der linearen Regression ergibt sich
\begin{align*}
  c &= \SI{17,0(8)e-3}{\per \s} \\
  d &= \SI{42(4)}{}.
\end{align*}

Mit der Gauß'schen Fehlerfortpflanzung \eqref{eqn:err2}
\begin{equation}
  \symup{\Delta} \tau_{\ce{^{104}Rh}} = \sqrt{\left(-\frac{\ln{(2)}}{c^2} \cdot \symup{\Delta}c \right)^2}
  \label{eqn:err2}
\end{equation}

ergibt sich die Halbwertszeit
\begin{equation*}
  \tau_{\ce{^{104}Rh}} = \frac{\ln(2)}{c} = \SI{40,7(19)}{\s}.
\end{equation*}

Der Theoriewert \cite{sample3} liegt bei $\tau_{\ce{^{104}Rh}, \text{theo}} = \SI{42}{\s}$, somit beträgt die Abweichung $3,10 \%$.

Für die gesamte Zerfallskurve gilt
\begin{equation*}
  N(t) = N_{\ce{^{104}Rh}} + N_{\ce{^{104i}Rh}}.
\end{equation*}

Diese ist in Graph \ref{fig:r3} halblogarithmisch aufgetragen und zusammen mit den bereits in Abbildung \ref{fig:r1} und \ref{fig:r2}
gezeigten Regressionen dargestellt.
\begin{figure}[H]
  \centering
  \includegraphics[width=\textwidth]{Plots/rhodges.pdf}
  \caption{Die gesamte Zerfallskurve bei halblogarithmischer Auftragung}
  \label{fig:r3}
\end{figure}

In Abbildung \ref{fig:r4} ist $N(t)$ nochmals aufgetragen, dieses mal aber bei linearer Auftragung.
\begin{figure}[H]
  \centering
  \includegraphics[width=\textwidth]{Plots/rhodgesnolog.pdf}
  \caption{Die gesamte Zerfallskurve}
  \label{fig:r4}
\end{figure}
