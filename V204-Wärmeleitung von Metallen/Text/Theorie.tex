
\section{Ziel}
In diesem Versuch soll die Wärmeleitung von Messing, Aluminium und Edelstahl untersucht werden.
\section{Theorie}
Ist die Temperatur eines Körpers ungleichmäßig in diesem verteilt, so findet ein Wärmetransport statt.
Dies kann durch Konvektion, also Wärmemitführung, Wärmestrahlung und Wärmeleitung geschehen,
wobei sich in diesem Versuch auf letzteres beschränkt wird.
Ebenso kann der Gitterbeitrag vernachlässigt werden, da im folgenden nur Metalle untersucht werden.
\\
Betrachtet wird zunächst ein Metallstab der Länge $L$ und Querschnittsfläche $A$,
dessen Enden eine Temperaturdifferenz $\Delta T$ aufweisen.
Aus dem 2. Hauptsatz der Thermodynamik folgt, dass nun Wärme vom wärmeren zum kälteren Ende fließt.
Es ergibt sich folgender Zusammenhang:
\begin{equation}
  \symup{d}Q=-\kappa A\frac{\symup{\partial}T}{\symup{\partial}x}\symup{d}t
  \label{eqn:dQ}
\end{equation}
\begin{center}
 \small {($Q=\text{Wärme}$,  $T=\text{Temperatur}$, $t=\text{Zeit}$, $ x \in [0, L]$ .)}
\end{center}
Das negative Vorzeichen folgt aus Konvention und zeigt, dass Temperatur vom wärmeren ins kältere Reservoir fließt.
Für die Wärmestromdichte $j_w$ gilt entsprechend
\begin{equation}
  j_w=-\kappa \frac{\symup{\partial}T}{\symup{\partial}x}
\end{equation}
Aus der Kontinuitätsgleichung
\begin{equation}
  \frac{\symup{\partial}\rho}{\symup{\partial}t} + \vec{\nabla} \cdot \vec{j} =0
\end{equation}
ergibt sich damit die Wärmeleitungsgleichung in $1D$:
\begin{equation}
  \frac{\symup{\partial}T}{\symup{\partial}t} = \frac{\kappa}{\rho c} \frac{\symup{\partial}^2T}{\symup{\partial}x^2 .}
  \label{eqn:WLG}
\end{equation}
\begin{center}
 \small {($c=\text{spezifische Wärmekapazität}$,  $m=\text{Masse}$, $\rho=\text{Dichte}$})
\end{center}
Der Faktor $\sigma_T=\frac{\kappa}{\rho c}$ wird dabei als Leitfähigkeit bezeichnet. Dieser gibt an, wie schnell der
Temperaturunterschied $\Delta T$ der beiden Stabenden ausgeglichen wird.
Die Lösung von \eqref{eqn:WLG} hängt von den gewählten Anfangsbedingungen ab.
Wird der Körper periodisch abgekühlt und erwärmt, so ergibt sich folgender Zusammenhang:
\begin{equation}
  T(x, t)=T_\text{max}\exp{\!\!\left[-\sqrt{\frac{\omega \rho c}{2\kappa}}\right]} \cos{\left(\omega t - \sqrt{\frac{\omega \rho c}{2\kappa}} x \right)}
\end{equation}
\begin{center}
 \small {(mit $\omega=\frac{2\pi}{T*}$,  $T*=\text{Periodendauer}$})
\end{center}
Für die Phasengeschwindigkeit $\nu=\frac{\omega}{k}$ ergibt sich damit:
\begin{equation}
  \nu =\sqrt{\frac{2\kappa \omega}{\rho c}}
\end{equation}
Mit den bereits bekannten Zusammenhängen folgt mit der Phase $\phi=\frac{2 \pi \Delta t}{T*}$ für die
Wärmeleitfähigkeit $\kappa$:
\begin{equation}
  \kappa=\frac{\rho c \left(\delta x\right)^2}{2 \Delta t \ln{\left(\frac{A_\text{nah}}{A_\text{fern}}\right)}}
  \label{eqn:kap}
\end{equation}
