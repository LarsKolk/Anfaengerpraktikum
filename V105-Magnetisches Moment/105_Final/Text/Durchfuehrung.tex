\section{Durchführung \cite{sample}}

\subsection{Apparatekonstanten bestimmen}

Zunächst müssen der Radius $r_K$ und die Masse $m_K$ der Billardkugel und die Länge des Stiels bestimmt werden.
Mithilfe dieser Werte wird anschließend Trägheitsmoment $J_K = \frac{2}{5} m_K \cdot r_K^2$ berechnet.
Außerdem muss auch die verschiebbare Masse $m$ gewogen werden.

\subsection{Bestimmung des magnetischen Moments mithilfe der Gravitation \label{sec:grav}}

Es wird eine Aluminiumstange mit einer verschiebbaren Masse $m$ in den Stiel der Kugel geschoben.
Dabei wirkt die Gravitationskraft $\vec{F_{\!\! g}} = m \cdot \vec{g}$ auf die Masse $m$,
die das Drehmoment $\vec{D_{\!\! g}} = m \cdot (\vec{r} \times \vec{g})$ ausübt.
In diesem Fall ist $\lvert \vec{r} \rvert$ der Abstand zwischen dem Zentrum der Billardkugel
und dem Schwerpunkt der Masse $m$.
Der Gravitation wirkt das Magnetfeld mit dem Drehmoment $\vec{D_{\!B}} = \vec{\mu} \times \vec{B}$ entgegen.
Im Gleichgewicht gilt:
\begin{align}
  m \cdot (\vec{r} \times \vec{g}) &= \vec{\mu} \times \vec{B} \\
   \implies r \cdot m \cdot g \cdot \sin{\varphi} &= \mu_{\symup{Dipol}} \cdot B \cdot \sin{\vartheta}.
\end{align}
Weil $\vec{g} \parallel \vec{B}$ ist, gilt $\varphi = \vartheta$. Daraus folgt:
\begin{equation}
  \mu_{\symup{Dipol}} \cdot B = m \cdot r \cdot g
  \label{eqn:mygrav}
\end{equation}
Zuerst wird die Kugel mit der Aluminiumstange und der Masse $m$ auf den Messingzylinder gestellt und
das Gebläse eingeschaltet.
Für einen gegebenen Abstand $r$ muss die Stromstärke $I$ und somit das Magnetfeld $B$ so eingestellt werden,
dass sich das System im Gleichgewicht befindet. Diese Messung soll für neun weitere Abstände wiederholt werden.

\subsection{Bestimmung des magnetischen Moments über die Schwingungsdauer \label{sec:schw}}

In diesem Teil des Experiments wird die Kugel in Schwingung versetzt, wobei sich die Kugel im
homogenen Magnetfeld wie ein harmonischer Oszillator verhält.
Die Bewegung der Kugel wird durch
\begin{equation}
  - \lvert \vec{\mu} \times \vec{B} \rvert = J_K \frac{d^2 \vartheta}{dt^2}
\end{equation}
beschrieben.
Durch Lösen der Differentialgleichung ergibt sich die Schwingungsdauer $T$:
\begin{equation}
  T^2 = \frac{4 \pi^2 J_K}{\mu_\symup{Dipol}}\frac{1}{B}.
  \label{eqn:sdauer}
\end{equation}
Die Kugel wird erneut auf den Messingzylinder gesetzt und das Gebläse eingeschaltet.
Bei einem eingestellten Magnetfeld wird die Kugel um einen kleinen Winkel ausgelenkt.
Es werden zehn Periodendauern gemessen und das Ergebnis gemittelt.
Dies wird für neun weitere Magnetfeldstärken wiederholt.

\subsection{Bestimmmung des magnetsichen Moments über die Präzession \label{sec:prae}}

Bei der Präzession bewegt sich die Drehachse eines rotierenden Körpers auf einem Kegelmantel
um die Drehimpulsachse $L$.
Die Bewegungsgleichung für eine Präzessionsbewegung lautet:
\begin{equation}
  \vec{\mu} \times \vec{B} = \frac{d \vec{L}_K}{dt}.
\end{equation}
Die dazugehörige Lösung ist die Präzessionsfrequenz:
\begin{equation}
  \Omega_P = \frac{\mu \cdot B}{\lvert \vec{L}_K \rvert}.
\end{equation}
Da $\Omega_P = \frac{1}{T_P}$ ist, gilt:
\begin{equation}
  \frac{1}{T_P} = \frac{\mu_{\symup{Dipol}}}{2 \pi L_K} B,
  \label{eqn:pfrq}
\end{equation}
wobei $L_K = J_K \cdot \omega$ ist.
\\
\\
Die Kugel wird auf das Luftkissen gesetzt und das Stroboskop auf eine
Frequenz zwischen $4 \mathrm{Hz}$ und $6 \mathrm{Hz}$ eingestellt.
Anschließend wird die Billardkugel in Rotation versetzt und aus der senkrechten Position ausgelenkt.
Die Rotationsfrequenz kann mithilfe des Stroboskops kontrolliert werden.
Auf dem Stiel der Billardkugel befindet sich eine weiße Markierung.
Rotiert die Kugel mit der am Stroboskop eingestellten Frequenz, erscheint die Markierung stationär.
Sobald dies der Fall ist wird das Magnetfeld eingeschaltet und dreimal die Präzessionsperiode $T_P$ gemessen.
Dies wird für insgesamt zehn Magnetfeldstärken $B$ wiederholt.
