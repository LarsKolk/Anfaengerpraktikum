\section{Theorie \cite{sample}}

\textbf{\underline{Ziel:}}
Die Bestimmung des magnetischen Moments eines Permanentmagneten auf drei unterschiedliche Arten.
\\
\\
Magnetische Dipole können entweder als Permanentmagneten oder als stromdurchflossene Leiter auftreten.
Das magnetische Moment einer stromdurchflossenen Leiterschleife ergibt sich aus:

\begin{equation}
  \vec{\mu} = I \cdot \vec{A},
\end{equation}
wobei $I$ die Stromstärke und $A$ die Querschnittfläche der Leiterschleife ist.
In einem homogenen Magnetfeld wirkt das Drehmoment
\begin{equation}
  \vec{D} = \vec{\mu} \times \vec{B}
\end{equation}
Um ein homogenes Magnetfeld zu erzeugen, wird ein Helmholtz-Spulenpaar eingesetzt.
Dieses besteht aus zwei gleichsinnig vom Strom $I$ durchflossenen Kreisspulen, die so positioniert werden,
dass ihre Achsen zusammenfallen.
Dabei ist der Abstand $d$ der Spulen gleich dem Spulenradius $R$.
\\
\\
Das Magnetfeld in der Mitte des Spulenpaares lässt sich mithilfe des Biot-Savart-Gesetzes
\begin{equation}
  d\vec{B} = \frac{\mu_0 I}{4 \pi} \frac{d\vec{s} \times \vec{r}}{r^3}
\end{equation}
bestimmen.
Das Magnetfeld für eine Spule mit einer Windung beträgt somit:
\begin{equation}
  \vec{B}(x) = \frac{\mu_0 I}{2} \frac{R^2}{(R^2 + x^2)^{\frac{3}{2}}} \vec{e_x}.
\end{equation}
Da im Experiment aber $d = 2x$ gilt, muss mit dem allgemeinen Fall
\begin{equation}
  B = \frac{\mu_0 I R^2}{(R^2 + x^2)^{\frac{3}{2}}}
  \label{eqn:bfeld}
\end{equation}
gerechnet werden.
