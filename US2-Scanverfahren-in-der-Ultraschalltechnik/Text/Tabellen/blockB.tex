\begin{table}[H]
   \centering
   \caption{Die mithilfe des B-Scans aufgenommenen und berechneten Werte}
   \label{tab:bscan}
   \begin{tabular} { c *5{S} }
 \toprule
 {Lochnummer} & {$d_\text{theo}\:/\: \mathrm{cm}$} & {$t_\text{oben}\:/\: \symup{\mu s}$} & {$t_\text{unten}\:/\: \symup{\mu s}$} &
 {$d\:/\: \mathrm{mm}$} & {Abweichung} \\
    \midrule
    3 & 0,6 & 45,93 & 10,74 & 6,65 & 10,83\% \\
    4 & 0,5 & 40,37 & 16,85 & 5,89 & 17,83\% \\
    5 & 0,4 & 34,81 & 23,15 & 4,88 & 22,01\% \\
    6 & 0,3 & 29,44 & 29,44 & 3,62 & 20,56\% \\
    7 & 0,2 & 23,52 & 35,19 & 3,87 & 93,47\% \\
    8 & 0,2 & 17,78 & 41,11 & 3,62 & 80,83\% \\
    9 & 0,2 & 11,85 & 47,04 & 3,62 & 80,83\% \\
    10 & 0,2 & 6,11 &  &  &  \\
    11 & 0,9 & 41,67 & 12,22 & 10,44 & 16,02\% \\
    \bottomrule
  \end{tabular}
\end{table}
