\section{Diskussion}

Die Kennlinien in Kapitel \ref{sec:kenn} folgen dem theoretisch erwartetem Trend. Da die Gleichspannungsquelle nicht weiter hochgeregelt
werden konnte, ist es nicht bei allen Kennlinien möglich einen Sättigungsstrom abzulesen.
Die aus der Heizleistung berechnete Temperatur weicht um $11,89 \%$ von der Temperatur ab, die sich aus der Messung des Anlaufstromgebietes ergibt.
Ein Grund dafür könnte sein, dass die Ströme im Nanoampere-Bereich liegen und aufgrunddessen eine empfindliche und somit auch
störungsanfällige Messvorrichtung verwendet werden muss. Desweiteren kommt es zu Schwankungen in den Messwerten wenn sich ein Objekt in der
Nähe der Leitung befindet.
Die Abweichung der errechneten Austrittsarbeit vom Literaturwert liegt mit $10,12 \%$ im Rahmen der Messungenauigkeiten.
