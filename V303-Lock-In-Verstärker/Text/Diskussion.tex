\section{Diskussion}

Im ersten und zweiten Teil des Experiments (s. Kapitel \ref{sec:knoise} und Kapitel \ref{sec:mnoise})
liegen die Abweichungen von der Theoriekurve bei unter $2 \%$.
Bei der Messung ohne Noise ist dies eigentlich auch zu erwarten, da das Signal nicht verrauscht ist.
Dass die Messung mit Noise auch so geringe Abweichungen liefert, lässt den Schluss zu,
dass der verwendete Lock-In-Verstärker gut funktioniert und die Störungen aus dem Signal herausfiltert.
Auffällig ist nur, dass Signal- und Referenzspannung von Beginn an Phasenverschoben sind.

Bei der Messung mit der LED (s. Kapitel \ref{sec:led}) liegt die Abweichung von $9,88 \%$ ebenfalls im Rahmen der
Messungenauigkeiten. Ein relevanter Störungsfaktor ist hier definitiv die Streuung des LED-Lichtes an
umliegenden Gegenständen und auch an den Experimentatoren selbst, was zu Verfälschungen der Ergebnisse
geführt haben kann.
