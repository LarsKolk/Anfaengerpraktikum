\section{Durchführung}
\subsection{Bestimmung der Wärmekapazität des Kalorimeters}
Zunächst wird Wasser in einem geeigneten Gefäß auf eine Temperatur $T_x$ erwärmt und dessen Masse $m_x$ mithilfe
der Schnellwaage bestimmt. Die Temperatur $T_x$ wird dabei mit einem Thermometer gemessen.
Ebenso wird das Kalorimeter mit einer zweiten Wassermenge befüllt, dessen Masse $m_y$ und
Temperatur $T_y$ auf gleiche Art bestimmt werden.
Daraufhin wird die erste Wassermenge in das Kalorimeter gegeben.
Es stellt sich infolgedessen eine Mischtemperatur $T'_m$ ein, die ebenfalls gemessen wird.

\subsection{Bestimmung der Wärmekapazitäten verschiedener Stoffe}
Das Kalorimeter wird zunächst mit einer Wassermenge der Masse $m_w$ befüllt.
Wenn die Temperatur der Wand des Kalorimeters mit der des Wassers übereinstimmt, wird die Temperatur $T_w$ gemessen.
Ebenso wird in einem geeigneten Gefäß Wasser erhitzt.
In dieses Wasser wird ein Material, dessen Masse $m_k$ vorher gemessen wurde, eingetaucht.
Wenn dieses eine Temperatur von mindestens \SI{80}{\celsius} erreicht hat, wird es dem Wasserbad entnommen
und in das Kalorimeter eingetaucht.
Die sich einstellende Temperatur im Kalorimeter wird gemessen.
Daraufhin wird der Probekörper entfernt und die Messung zwei weitere Male für dieses Material durchgeführt.
Die Messung soll für 3 Materiale durchgeführt werden.
