\section{Diskussion}

Die Abweichungen sind mit $44,91 \%$ in Kapitel \ref{sec:empf} und $15,05 \%$ in Kapitel \ref{sec:spez} relativ groß.
Dies kann mehrere Gründe haben. Der Elektronenstrahl muss jedes mal per Hand neu fokussiert werden, was zu unterschiedlich
großen Flecken auf dem Leuchtschirm führt wodurch es schwierig ist den Strahl immer um den denselben Abstand zu verschieben.
Desweiteren kam es bei der Messung in Kapitel \ref{sec:empf} zu Schwankungen bei den vom Voltmeter angezeigten Werten.
Auch ist der Abstand zwischen den Kondensatorplatten nicht überalll gleich groß, wodurch dieser zur Berechnung des Theoriewertes
gemittelt werden muss.
Bei der Bestimmung des Erdmagnetfeldes kommt als zusätzliche Schwierigkeit die Handhabung des Deklinatorium-Inklinatoriums hinzu, da
sich die Nadel nicht richtig ausgerichtet hat.
