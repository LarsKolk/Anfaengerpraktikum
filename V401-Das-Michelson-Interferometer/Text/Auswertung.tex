\section{Auswertung}

\subsection{Bestimmung der Wellenlänge des verwendeten Lasers \label{sec:lam}}

In Tabelle \ref{tab:lam} befinden sich die aufgenommenen Messwerte.
\begin{table}[H]
   \centering
   \caption{Aufgenommene Messwerte und berechneten Werte für die Wellenlänge des verwendeten Lasers}
   \label{tab:lam}
   \begin{tabular} { c c c c }
 \toprule
 {$\symup{\Delta}d_\text{abgelesen}\:/\: \mathrm{mm}$} & {$\symup{\Delta}d\:/\: \mathrm{mm}$} & {$z$} & {$\lambda\:/\: \mathrm{nm}$} \\
    \midrule
    4,80 & 0,96 & 3020 & 633,61 \\
    4,80 & 0,96 & 3043 & 628,82 \\
    5,24 & 1,04 & 3008 & 694,45 \\
    5,04 & 1,00 & 3001 & 669,50 \\
    4,25 & 0,85 & 3000 & 564,75 \\
    4,18 & 0,83 & 3001 & 555,26 \\
    4,96 & 0,99 & 3000 & 659,09 \\
    4,79 & 0,95 & 3000 & 636,50 \\
    5,10 & 1,01 & 3000 & 673,80 \\
    5,06 & 1,00 & 2999 & 668,74 \\
    \bottomrule
  \end{tabular}
\end{table}


Die von der Mikrometerschraube abgelesene Spiegelverschiebung $\symup{\Delta}d_\text{abgelesen}$ muss mithilfe der Übersetzung $Ü$
in die tatsächliche Verschiebung $\symup{\Delta}d$ umgerechnet werden. Für die letzten beiden Werte beträgt die Übersetzung
$\ddot{U} = \frac{1}{5,046}$, für den Rest beträgt sie $\ddot{U} = \frac{1}{5,017}$.

Die Werte für $\lambda$ ergeben sich aus Gleichung \eqref{eqn:lam}
\begin{equation}
  \lambda = \frac{2 \symup{\Delta}d}{z}.
  \label{eqn:lam}
\end{equation}

Der Mittelwert und die Standardabweichung ergeben sich aus den Gleichungen \eqref{eqn:mit} und \eqref{eqn:sta}
\begin{align}
  \bar{N} &= \frac{1}{n} \sum_{i=1}^{n} N_i
  \label{eqn:mit} \\
  \sigma_{\bar{N}} &= \sqrt{\frac{1}{n (n - 1)} \sum_{i=1}^{n} (N_i - \bar{N})^2}.
  \label{eqn:sta}
\end{align}

Im Mittel beträgt $\lambda$ somit
\begin{equation*}
  \bar{\lambda} = \SI{638(15)}{\nm}.
\end{equation*}

Der Theoriewert beträgt $\lambda_\text{theo} = \SI{635}{\nm}$. Die relative Abweichung beträgt $\SI{0,54}{\%}$, dies entspricht $\SI{0,24}{}$
Fehlerintervallen.

\subsection{Bestimmung des Brechungsindex von Luft \label{sec:n}}

In Tabelle \ref{tab:n} befinden sich die aufgenommenen Messwerte.
\begin{table}[H]
   \centering
   \caption{Aufgenommene Messwerte und berechente Werte für den Brechungsindex von Luft}
   \label{tab:n}
   \begin{tabular} { c c c }
 \toprule
 {$\symup{\Delta}p\:/\: \mathrm{bar}$} & {$z$} & {$n$} \\
    \midrule
    0,66 & 27 & 1,00028 \\
    0,80 & 34 & 1,00029 \\
    0,80 & 33 & 1,00029 \\
    0,80 & 31 & 1,00027 \\
    0,80 & 32 & 1,00028 \\
    0,76 & 30 & 1,00027 \\
    0,80 & 34 & 1,00029 \\
    0,76 & 30 & 1,00027 \\
    0,80 & 34 & 1,00029 \\
    0,78 & 32 & 1,00028 \\
    \bottomrule
  \end{tabular}
\end{table}


Die Werte für $n$ unter Normalbedingungen ergeben sich aus Gleichung \eqref{eqn:n}
\begin{align}
  n = 1 + \frac{z \lambda}{2 b} \frac{T}{T_0} \frac{p_0}{\symup{\Delta}p}.
  \label{eqn:n}
\end{align}
\begin{center}
  \small{($b = \SI{50}{\mm} \: \hat{=} \: \text{Größe der Messzelle}$, $p_0 = \SI{1,0132}{\bar}$, $T_0 = \SI{273,15}{\K}$, $T = \SI{294,15}{\K}$)}
\end{center}

Der Mittelwert und die Standardabweichung nach Gleichung \eqref{eqn:mit} und \eqref{eqn:sta} beträgt
\begin{equation*}
  \bar{n} = \SI{1,000283(3)}{}.
\end{equation*}

Der Literaturwert \cite{sample2} für $\lambda = \SI{635}{\nm}$ beträgt $n_\text{theo} = \SI{1,00028}{}$. Die relative Abweichung liegt
bei $\SI{0,0003}{\%}$. Der Mittelwert weicht um $\SI{0,97}{}$ Fehlerintervalle ab.
