\section{Auswertung}

\subsection{Kapazität des Kalorimeters}

Die aufgenommenen Messwerte sind
\begin{align*}
  m_x &= \SI{300}{\g}       \\
  m_y &= \SI{300}{\g} \\
  T_x &= \SI{293,86}{\K} \\
  T_y &= \SI{355,75}{\K} \\
  T_m &= \SI{324,05}{\K}.
\end{align*}

Die spezifische Wärmekapazität von Wasser beträgt $c_w = \SI{4,18}{\joule \per \gram \kelvin}$ \cite[159]{sample1}.
Mit Gleichung \eqref{eqn:kal} ergibt sich für die Kapazität des Kalorimeters
\begin{equation*}
  c_g m_g = \SI{62,28}{\joule \per \kelvin}.
\end{equation*}

\subsection{Ermittlung der spezifischen Wärmekapazität}

\subsubsection{Graphit}

Die aufgenommenen Messwerte befinden sich in Tabelle \ref{tab:graphit}. Die Masse des Probenkörpers beträgt hier
$m_g = \SI{97,75}{\g}$.
Die Werte für die spezifische Wärmekapazität $c_g$ ergeben sich aus Gleichung \eqref{eqn:ck}.
\begin{table}[H]
  \centering
  \caption{Werte zur Bestimmung der spezifischen Wärmekapazität von Graphit}
  \label{tab:graphit}
  \begin{tabular}{c c c c}
    \toprule
       & {1. Messung} & {2. Messung} & {3. Messung}\\
    \midrule
       $m_W \:/\: \mathrm{g}$ & 600 & 600 & 600 \\
       $T_W \:/\: \mathrm{K}$ & 295,45 & 295,45 & 298,75 \\
       $T_g \:/\: \mathrm{K}$ & 355,95 & 365,45 & 360,65 \\
       $T_m \:/\: \mathrm{K}$ & 295,75 & 297,65 & 299,55 \\
       $c_g \:/\: \mathrm{\frac{J}{g K}}$ & 0,131 & 0,853 & 0,344 \\
    \bottomrule
  \end{tabular}
\end{table}

Der Mittelwert ergibt sich aus
\begin{equation}
  \bar{x} = \frac{1}{N} \sum_{i=1}^{N} x_i
  \label{eqn:mit}
\end{equation}

und die Standardabweichung aus
\begin{equation}
  \Delta \bar{x} = \sqrt{\frac{1}{N (N - 1)} \sum_{i=1}^{N} (x_i - \bar{x})^2}.
  \label{eqn:sta}
\end{equation}

Angewendet ergibt das
\begin{align*}
  \bar{c}_g &= \frac{c_{g1}+c_{g2}+c_{g3}}{3} \\
  \Delta \bar{c}_g &= \sqrt{\frac{1}{6} \left((c_{g1}-\bar{c}_g)^2 + (c_{g2}-\bar{c}_g)^2 + (c_{g3}-\bar{c}_g)^2 \right)}.
\end{align*}

Somit beträgt die gemittelte Wärmekapazität
\begin{equation*}
  \bar{c}_g = \SI{0,443(214)}{\joule \per \gram \per \kelvin}.
\end{equation*}

Der Theoriewert beträgt $c_\text{theo} = \SI{0,715}{\joule \per \gram \per \kelvin}$ (s. \cite{sample2}).
Daher beträgt die Abweichung $38,06 \%$.

\subsubsection{Blei}
Die aufgenommenen Messwerte befinden sich in Tabelle \ref{tab:blei}.
Die spezifischen Wärmekapazitäten $c_b$ ergeben sich aus Gleichung \eqref{eqn:ck}.
Der Probenkörper hat die Masse $m_b = \SI{535,33}{\g}$.
\begin{table}[H]
  \centering
  \caption{Werte zur Bestimmung der spezifischen Wärmekapazität von Blei}
  \label{tab:blei}
  \begin{tabular}{c c c c}
    \toprule
       & {1. Messung} & {2. Messung} & {3. Messung}\\
    \midrule
      $m_W \:/\: \mathrm{g}$ & 600 & 600 & 600 \\
      $T_W \:/\: \mathrm{K}$ & 299,15 & 299,75 & 303,45 \\
      $T_b \:/\: \mathrm{K}$ & 357,55 & 366,15 & 367,35 \\
      $T_m \:/\: \mathrm{K}$ & 301,45 & 302,55 & 303,95 \\
      $c_b \:/\: \mathrm{\frac{J}{g K}}$ & 0,197 & 0,211 & 0,038 \\
    \bottomrule
  \end{tabular}
\end{table}

Die Mittelung mithilfe von \eqref{eqn:mit} und \eqref{eqn:sta} ergibt
\begin{align*}
  \bar{c}_b &= \frac{c_{b1}+c_{b2}+c_{b3}}{3} \\
  \Delta \bar{c}_b &= \sqrt{\frac{1}{6} \left((c_{b1}-\bar{c}_b)^2 + (c_{b2}-\bar{c}_b)^2 + (c_{b3}-\bar{c}_b)^2 \right)}.
\end{align*}

Die spezifische Wärmekapazität beträgt
\begin{equation*}
  \bar{c}_b = \SI{0,149(056)}{\joule \per \gram \per \kelvin}.
\end{equation*}

Der Literaturwert ist $c_\text{theo} = \SI{0,129}{\joule \per \gram \per \kelvin}$ (s. \cite{sample2}).
Die Abweichung beträgt $15,27 \%$.

\subsubsection{Aluminium}
Die aufgenommenen Messwerte befinden sich in Tabelle \ref{tab:alu}. Der Probenkörper wiegt $\SI{106,58}{\g}$.
Die spezifischen Wärmekapazitäten $c_a$ ergeben sich aus Gleichung \eqref{eqn:ck}.
\begin{table}[H]
  \centering
  \caption{Werte zur Bestimmung der spezifischen Wärmekapazität von Aluminium}
  \label{tab:alu}
  \begin{tabular}{c c c c}
    \toprule
       & {1. Messung} & {2. Messung} & {3. Messung}\\
    \midrule
      $m_W \:/\: \mathrm{g}$ & 600 & 600 & 600 \\
      $T_W \:/\: \mathrm{K}$ & 303,55 & 305,75 & 307,15 \\
      $T_a \:/\: \mathrm{K}$ & 363,65 & 360,15 & 356,15 \\
      $T_m \:/\: \mathrm{K}$ & 304,65 & 307,35 & 308,35 \\
      $c_a \:/\: \mathrm{\frac{J}{g K}}$ & 0,450 & 0,731 & 0,605 \\
    \bottomrule
  \end{tabular}
\end{table}

Mithilfe der Gleichungen \eqref{eqn:mit} und \eqref{eqn:sta} ergibt sich die Mittelung
\begin{align*}
  \bar{c}_a &= \frac{c_{a1}+c_{a2}+c_{a3}}{3} \\
  \Delta \bar{c}_a &= \sqrt{\frac{1}{6} \left((c_{a1}-\bar{c}_a)^2 + (c_{a2}-\bar{c}_a)^2 + (c_{a3}-\bar{c}_a)^2 \right)}.
\end{align*}

Für die spezifische Wä Wärmekapazität ergibt sich somit
\begin{equation*}
  \bar{c}_a = \SI{0,595(081)}{\joule \per \gram \per \kelvin}.
\end{equation*}

Der Literaturwert beträgt $c_\text{theo} = \SI{0,895}{\joule \per \gram \per \kelvin}$ (s. \cite{sample2}).
Die Abweichung beträgt somit $33,49 \%$.

\subsection{Berechnung der Molwärme \label{sec:mol}}

Die Molwärme lässt sich mithilfe von Gleichung \eqref{eqn:molw} bestimmen.
Mithilfe von
\begin{align*}
  \bar{T}_m &= \frac{T_{m1}+T_{m2}+T_{m3}}{3} \\
  \Delta \bar{T}_m &= \sqrt{\frac{1}{6} \left((T_{m1}-\bar{T}_m)^2 + (T_{m2}-\bar{T}_m)^2 + (T_{m3}-\bar{T}_m)^2 \right)}.
\end{align*}
werden die Mischtemperaturen zu
\begin{align*}
  \bar{T}_{mg} &= \SI{297,65(110)}{\K} \\
  \bar{T}_{mb} &= \SI{302,65(72)}{\K} \\
  \bar{T}_{ma} &= \SI{306,78(111)}{\K}
\end{align*}
gemittelt.
Die Werte für $\rho$, $M$, $\alpha$ und $\kappa$ befinden sich in der Versuchsanleitung \cite[159]{sample1}.
Die Ergebnisse sind in Tabelle \ref{tab:mol} dargestellt.
Die Molwärme nach Dulong-Petit beträgt
\begin{equation}
  C_V = 3 R = \SI{24,942}{\joule \per \mol \per \K}.
\end{equation}

\begin{table}[H]
 \centering
 \caption{Molwärmen der einzelnen Probenkörper und die Abweichungen vom Dulong-Petitschen Gesetz}
 \label{tab:mol}
 \begin{tabular}{c c c c}
   \toprule
      & {Graphit} & {Blei} & {Aluminium} \\
   \midrule
     $C_V \:/\: \mathrm{\frac{J}{mol K}}$ & 5,284 \pm 2,571 & 29,053 \pm 11,519 & 14,929 \pm 2,193 \\
     Abweichung & 78,82 \% & 16,48 \% & 40,15 \% \\
   \bottomrule
 \end{tabular}
\end{table}

Die Abweichungen von $C_V$ werden mithilfe der Gauß'schen Fehlerfortpflanzung
\begin{equation*}
  \Delta C_V = \sqrt{\left(M \Delta c \right)^2 + \left(-9 \alpha^2 \kappa \frac{M}{\rho} \Delta T_m \right)^2}
\end{equation*}
berechnet.
