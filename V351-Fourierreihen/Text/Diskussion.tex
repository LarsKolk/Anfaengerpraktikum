\section{Diskussion}

Bei der Analyse der Rechteck- und Sägezahn-Spannung sind die Abweichungen mit $13,8 \%$ und $0,9 \%$
relativ klein und liegen im Rahmen der Messungenauigkeiten.
Allerdings ist die Abweichung bei der Dreieck-Spannung mit $54,6 \%$ ziemlich groß.
Ursache dafür kann der schnelle Abfall mit $\sfrac{1}{n^2}$ sein, der eine genaue Amplitudenmessung stark
erschwert hat.

Bei der Synthese ließen sich in allen drei Fällen relativ gute Ergebnisse erzielen.
Die Schwierigkeiten bei diesem Teil des Experiments, liegen darin, dass sich aufgrund der Schwankungen in den
Oberwellen-Spannungen die gewünschten Amplituden nicht genau einstellen lassen.
Außerdem war problematisch, dass die geringste einstellbare Spannung bei etwa $\SI{15}{\V}$ lag
und es bei den schnell fallenden Amplituden der Dreieck-Spannung nicht möglich war mehr als
Grundfrequenz und eine Oberwelle einzustellen.
Allerdings ist das Ergebnis gerade aufgrund des starken Abfalls, wodurch die weiteren Oberwellen nur einen
geringen Einfluss haben, ziemlich gut.
