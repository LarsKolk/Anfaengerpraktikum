\section{Durchführung}
\label{sec:Durchführung}
%Da Luft den Schall zu stark absorbiert, wird als Kopplungsmittel in allen Versuchsteilen, in denen das Impuls-Echo-Verfahren genutzt wird, bidestilliertes Wasser verwendet.
%Bei der Durchschallungs-Methode wird Koppelgel verwendet.

%\subsection{Ausmessung der Acrylzylinder}
%Im ersten Versuchsteil wird zunächst die Höhe eines Acrylzylinder bestimmt.
%Dieser wird nun auf ein weißes Tuch gelegt und eine Messsonde auf diesem platziert.
%Nun wird mithilfe des Messprogramms die Zeit $\Delta t$ gemessen, die ein Schallimpuls benötigt, um vom Sender
%wieder zum zurück zu diesem zu kommen. Als Kopplungsmittel wird bidestilliertes Wasser verwendet.
%Mit den nun bekannten Daten wird der Wert für die Schallgeschwindigkeit im Acrylzylinder bestimmt.
%Diese wird daraufhin ins Messprogramm eingetragen, damit eine Tiefenmessung durchgeführt werden kann. Das Ergebnis wird mit der Messung verglichen.
%Dies wird für $6$ weitere zylinder wiederholt.
%Wir haben das aber gemacht :(

\subsection{Messung der Dämpfung mittels Impuls-Echo-Verfahren}
Ein Acrylzylinder wird auf ein Papiertuch gestellt und an eine $\SI{2}{\mega\hertz}$-Sonde gekoppelt. Als Kopplungsmittel wird bidestilliertes Wasser verwendet.
Die Sonde wird mit einem  Echoskop im \textit{REFLEC.}-Modus verbunden. Abstand und Betrag der ersten beiden Peaks werden mithilfe des Messprogramms bestimmt.
Die Messung wird für $6$ weitere Zylinder wiederholt.

\subsection{Schallgeschwindigkeitsbestimmung}
\subsubsection{Durchschallungs-Verfahren}
Ein bereits abgemessender Acrylzylinder wird waagerecht in eine geeignete Halterung gelegt.
An beide Enden wird eine $\SI{2}{\mega\hertz}$-Sonde angelegt und als Kopplungsmittel Koppelgel verwendet.
Die Sonden werden an das Echoskop angeschlossen. Dieses muss sich dabei im \textit{TRANS.}-Modus befinden. Mithilfe des Messprogramms wird die Zeitdifferenz der Impulse gemessen.
Dies wird für $6$ weitere Zylinder wiederholt.
\subsubsection{Impuls-Echo-Verfahren}
Ein Acrylzylinder wird auf ein Papiertuch gestellt und an eine $\SI{2}{\mega\hertz}$-Sonde gekoppelt. Als Kopplungsmittel wird bidestilliertes Wasser verwendet.
Mithilfe eines A-Scans wird die Laufzeit des Schalls bestimmt, indem im Auswertungsprogramm die Zeitdifferenz zwischen dem Sende- und Echoimpuls bestimmt wird.
Der Vorgang wird für $6$ Zylinder wiederholt.
%hier mache ich nachher weiter

\subsection{Spektrale Analyse und Cepstrum}
Zwei Acrylscheiben mit verschiedener Dicke $d$ werden übereinander gelegt. Auf diese wird nun ein $\SI{4e-2}{\metre}$ langer Acrylzylinder gestellt, welcher
mit einer $\SI{2}{\mega\hertz}$-Sonde verbunden wird.
Zur Kopplung der Elemente wird bidestilliertes Wasser verwendet.
Mithilfe des Auswertungsprogramm wird eine Mehrfachreflexion aufgenommen. Dazu wird die Verstärkung am Echoskops so eingestellt, dass 3 Mehrfachreflexionen zu sehen sind.
Die zeitlichen Abstände zwischen den Echos werden mithilfe des Auswertungsprogramms ermittelt und das Cepstrum der Sonde mit der FFT-Funktion aufgezeichnet.

\subsection{Untersuchung eines Augenmodels}
In diesem Versuchsteil wird eine $\SI{2}{\mega\hertz}$-Sonde an die Hornhaut eines Augenmodells gekoppelt. Als Kopplungsmittel wird dabei Koppelgel verwendet.
Der Einschallwinkel wird solange leicht verändert, bis ein Echo an der Rückwand der Retina zu sehen ist.
Mithilfe eines A-Scans wird der Abstand von Hornhaut, Iris, Linse und Retina bestimmt.
%wird nachher ergänzt
